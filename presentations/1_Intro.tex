% Options for packages loaded elsewhere
\PassOptionsToPackage{unicode=true}{hyperref}
\PassOptionsToPackage{hyphens}{url}
%
\documentclass[
  ignorenonframetext,
]{beamer}
\usepackage{pgfpages}
\setbeamertemplate{caption}[numbered]
\setbeamertemplate{caption label separator}{: }
\setbeamercolor{caption name}{fg=normal text.fg}
\beamertemplatenavigationsymbolsempty
% Prevent slide breaks in the middle of a paragraph
\widowpenalties 1 10000
\raggedbottom
\setbeamertemplate{part page}{
  \centering
  \begin{beamercolorbox}[sep=16pt,center]{part title}
    \usebeamerfont{part title}\insertpart\par
  \end{beamercolorbox}
}
\setbeamertemplate{section page}{
  \centering
  \begin{beamercolorbox}[sep=12pt,center]{part title}
    \usebeamerfont{section title}\insertsection\par
  \end{beamercolorbox}
}
\setbeamertemplate{subsection page}{
  \centering
  \begin{beamercolorbox}[sep=8pt,center]{part title}
    \usebeamerfont{subsection title}\insertsubsection\par
  \end{beamercolorbox}
}
\AtBeginPart{
  \frame{\partpage}
}
\AtBeginSection{
  \ifbibliography
  \else
    \frame{\sectionpage}
  \fi
}
\AtBeginSubsection{
  \frame{\subsectionpage}
}
\usepackage{lmodern}
\usepackage{amssymb,amsmath}
\usepackage{ifxetex,ifluatex}
\ifnum 0\ifxetex 1\fi\ifluatex 1\fi=0 % if pdftex
  \usepackage[T1]{fontenc}
  \usepackage[utf8]{inputenc}
  \usepackage{textcomp} % provides euro and other symbols
\else % if luatex or xelatex
  \usepackage{unicode-math}
  \defaultfontfeatures{Scale=MatchLowercase}
  \defaultfontfeatures[\rmfamily]{Ligatures=TeX,Scale=1}
\fi
% Use upquote if available, for straight quotes in verbatim environments
\IfFileExists{upquote.sty}{\usepackage{upquote}}{}
\IfFileExists{microtype.sty}{% use microtype if available
  \usepackage[]{microtype}
  \UseMicrotypeSet[protrusion]{basicmath} % disable protrusion for tt fonts
}{}
\makeatletter
\@ifundefined{KOMAClassName}{% if non-KOMA class
  \IfFileExists{parskip.sty}{%
    \usepackage{parskip}
  }{% else
    \setlength{\parindent}{0pt}
    \setlength{\parskip}{6pt plus 2pt minus 1pt}}
}{% if KOMA class
  \KOMAoptions{parskip=half}}
\makeatother
\usepackage{xcolor}
\IfFileExists{xurl.sty}{\usepackage{xurl}}{} % add URL line breaks if available
\IfFileExists{bookmark.sty}{\usepackage{bookmark}}{\usepackage{hyperref}}
\hypersetup{
  pdftitle={Intro to CASA Coding Group},
  hidelinks,
}
\urlstyle{same} % disable monospaced font for URLs
\newif\ifbibliography
\usepackage{longtable,booktabs}
\usepackage{caption}
% Make caption package work with longtable
\makeatletter
\def\fnum@table{\tablename~\thetable}
\makeatother
\setlength{\emergencystretch}{3em} % prevent overfull lines
\providecommand{\tightlist}{%
  \setlength{\itemsep}{0pt}\setlength{\parskip}{0pt}}
\setcounter{secnumdepth}{-\maxdimen} % remove section numbering

% Set default figure placement to htbp
\makeatletter
\def\fps@figure{htbp}
\makeatother


\title{Intro to CASA Coding Group}
\author{}
\date{\vspace{-2.5em}25/02/2020}

\begin{document}
\frame{\titlepage}

\begin{frame}{Outline for today}
\protect\hypertarget{outline-for-today}{}

\begin{enumerate}
\tightlist
\item
  Purpose of group
\item
  Our tools: Languages/software we will learn
\item
  Examples of coding applications for speech
\item
  Download R, RStudio, Praat
\item
  Intro to RStudio
\item
  Topics for the term
\end{enumerate}

\end{frame}

\hypertarget{purpose-of-this-group}{%
\section{Purpose of this group}\label{purpose-of-this-group}}

\begin{frame}{Purpose of this group}
\protect\hypertarget{purpose-of-this-group-1}{}

\begin{itemize}
\tightlist
\item
  Develop skills that make it easier to do our job well as speech
  researchers
\item
  Create a community that comes together to make it easier to learn this
  stuff
\end{itemize}

\end{frame}

\begin{frame}{Purpose of this group}
\protect\hypertarget{purpose-of-this-group-2}{}

\textbf{Computer Coding}: Writing something in a language a computer can
understand in order to tell the computer to do a specific thing or set
of things.

Why bother telling a computer what to do when we can just do it
ourselves?

\end{frame}

\begin{frame}{Purpose of this group}
\protect\hypertarget{purpose-of-this-group-3}{}

\textbf{\emph{Why bother telling a computer what to do when we can just
do it ourselves?}}

\begin{itemize}
\tightlist
\item
  Automate repetitive tasks like\ldots{}

  \begin{itemize}
  \tightlist
  \item
    opening/closing/saving files
  \end{itemize}
\item
  Minimize human error in data preparation

  \begin{itemize}
  \tightlist
  \item
    Renaming things in a spreadsheet
  \end{itemize}
\item
  Keep a careful log of how we did our analyses

  \begin{itemize}
  \tightlist
  \item
    Code = instructions
  \end{itemize}
\item
  ``Reproducible research''
\item
  Fun! (seriously!)
\end{itemize}

\end{frame}

\hypertarget{our-tools-logistics}{%
\section{Our tools: Logistics}\label{our-tools-logistics}}

\begin{frame}{Project Website}
\protect\hypertarget{project-website}{}

\begin{itemize}
\tightlist
\item
  Website: \url{https://casa-lab.com/coding-group/}
\item
  Slack channel: \url{https://casa-lab-ub.slack.com/}

  \begin{itemize}
  \tightlist
  \item
    \href{https://join.slack.com/t/casa-lab-ub/shared_invite/enQtOTU2OTcwMzg3MDQzLTk4NmM0MjhlYTAyY2JkODhkNDZkYmE5MTc3M2VkZTEyNDg1YjhmMmM0ZmEwYjlkYmI5NTMyZjYxYjk5MDNmMWQ}{Invite
    to join Slack channel} (I will send this out via email after today)
  \end{itemize}
\end{itemize}

\end{frame}

\hypertarget{our-tools-languages-software}{%
\section{Our tools: Languages \&
Software}\label{our-tools-languages-software}}

\begin{frame}{First: Some terminology}
\protect\hypertarget{first-some-terminology}{}

\textbf{Coding}: Writing in a language a computer can understand

\textbf{Scripting}: A type of coding that tells a specific program
exactly what actions to take

\textbf{Programming}: Writing code that serves to actually create
another program (an app, software, etc)

\textbf{Scripts}: Text files containing code.

\begin{itemize}
\tightlist
\item
  Scripting, coding, and programming are sometimes used interchangeably
\end{itemize}

\end{frame}

\begin{frame}[fragile]{First: Some terminology}
\protect\hypertarget{first-some-terminology-1}{}

\textbf{Functions}: A certain named format of code that outlines a
procedure. Often this allows several lines of code to be executed with a
single line of code (by using the name of the function)

\begin{itemize}
\tightlist
\item
  For example, in Excel, you may use functions like \texttt{=sum(2,2)}.
  \texttt{sum()} is the function that takes input (in this case,
  numbers), and performs an a specific action (adds them).
\end{itemize}

\textbf{Calling}: Invoke a function by using the name of the function
and specifying parameters.

\begin{itemize}
\tightlist
\item
  For example: I ``call the sum function'' when I type it out with its
  inputs and execute it in excel.
\end{itemize}

\end{frame}

\begin{frame}{Our tools}
\protect\hypertarget{our-tools}{}

\begin{enumerate}
\tightlist
\item
  R and R Studio
\item
  Praat
\end{enumerate}

\end{frame}

\begin{frame}{}
\protect\hypertarget{section}{}

\begin{itemize}
\tightlist
\item
  ``R is a free software environment for statistical computing and
  graphics.''
\item
  \href{https://www.r-project.org/}{Download here}
\end{itemize}

\end{frame}

\begin{frame}{}
\protect\hypertarget{section-1}{}

\begin{itemize}
\tightlist
\item
  RStudio is a handy interface that helps you use R.
\item
  \href{https://rstudio.com/products/rstudio/download/}{Download Desktop
  version}
\end{itemize}

\end{frame}

\begin{frame}{ Praat}
\protect\hypertarget{praat}{}

\begin{itemize}
\tightlist
\item
  ``Doing phonetics by computer'': Praat is a powerful software program
  that also has its own specialized language for writing scripts
\item
  Praat = ``Speech'' in Dutch
\item
  Looks like it hasn't been updated since 1995 but it has and it's great
\item
  ``World's worst programming language''

  \begin{itemize}
  \tightlist
  \item
    \emph{don't let the haters get you down}
  \end{itemize}
\item
  \href{http://www.fon.hum.uva.nl/praat/}{Download here}
\end{itemize}

\end{frame}

\hypertarget{examples-of-coding-applications-for-speech-research}{%
\section{Examples of coding applications for speech
research}\label{examples-of-coding-applications-for-speech-research}}

\begin{frame}{1. Data preparation in }
\protect\hypertarget{data-preparation-in}{}

\textbf{Example: Starbucks data}

\begin{enumerate}
\tightlist
\item
  Start with a data set you have in Excel
\item
  ``Read'' it into R
\item
  Do things to it like\ldots{}
\end{enumerate}

\begin{itemize}
\tightlist
\item
  Instantly calculate means values
\end{itemize}

\textbf{Let's look together}

\end{frame}

\begin{frame}{1. Data preparation: Raw data}
\protect\hypertarget{data-preparation-raw-data}{}

\end{frame}

\begin{frame}[fragile]{1. Data preparation: Data prep script}
\protect\hypertarget{data-preparation-data-prep-script}{}

\texttt{1\_prep\_data.R}

\begin{verbatim}
###################################################
# Helper script for analyzing Starbucks drink data
###################################################

# Setup ----

# Load packages that contain functions we will use
library(tidyverse)
library(plyr)

# Load data ----
starbucks <- read.csv("1_materials/starbucks_drinkMenu_expanded.csv")
\end{verbatim}

\end{frame}

\begin{frame}[fragile]{1. Data preparation: Data prep script
(Continued)}
\protect\hypertarget{data-preparation-data-prep-script-continued}{}

\texttt{1\_prep\_data.R}

\begin{verbatim}
# Create new columns ----
# Create a "caffeine" column that is numeric
starbucks <- starbucks %>%
        mutate(caffeine = revalue(caffeine,replace = c(
                                  "varies" = NA, "Varies" = NA)),
               caffeine_num = as.numeric(as.character(caffeine)))

# Is caffeine content over 100 mg? If so, label it "YES", otherwise, "NO"
starbucks <- starbucks %>%
        mutate(too_much_caffeine = ifelse(caffeine_num > 100, "YES", "NO"))

starbucks %>% select(caffeine_num, too_much_caffeine) %>% head()
\end{verbatim}

\end{frame}

\begin{frame}[fragile]{2. Data visualization in }
\protect\hypertarget{data-visualization-in}{}

\texttt{2\_figures.R}

\end{frame}

\begin{frame}[fragile]{3. Writing in }
\protect\hypertarget{writing-in}{}

Using \texttt{R\ Markdown} to write:

\begin{itemize}
\tightlist
\item
  Notes \& reports
\item
  Papers, articles, theses
\item
  Presentations (like this one!)
\item
  Websites, blog posts!
\end{itemize}

\texttt{R\ Markdown} allows you to incorporate \emph{code} AND regular
text using simple ``markdown'' syntax (more on that later).

\end{frame}

\begin{frame}{4. Automating repetitive tasks in }
\protect\hypertarget{automating-repetitive-tasks-in}{}

For example\ldots{}

\begin{itemize}
\tightlist
\item
  \href{https://github.com/thealk/PraatScripts/blob/master/Editing_audio_stimuli/createTextGrid.praat}{Automatically
  create TextGrids for all .wav files in a directory}
\item
  \href{https://github.com/thealk/PraatScripts/blob/master/Praat_scripting_tutorial/3_adjustBoundaries.praat}{Automatically
  adjust Praat TextGrid boundaries for all files in a directory}
\end{itemize}

\end{frame}

\begin{frame}{5. Running experiments in }
\protect\hypertarget{running-experiments-in}{}

\href{https://theaknowles.com/post/measuring-speech-intelligibility-in-praat-part1/}{Intelligibility
experiment in Praat}

\end{frame}

\hypertarget{intro-to-rstudio}{%
\section{Intro to RStudio}\label{intro-to-rstudio}}

\begin{frame}{RStudio layout}
\protect\hypertarget{rstudio-layout}{}

\end{frame}

\begin{frame}{RStudio layout: Source pane}
\protect\hypertarget{rstudio-layout-source-pane}{}

This is where you'll edit and run your scripts.

\end{frame}

\begin{frame}{RStudio layout: Console pane}
\protect\hypertarget{rstudio-layout-console-pane}{}

This is where code, error messages, warnings, etc. show up when you run
code

\end{frame}

\begin{frame}[fragile]{RStudio layout: Files/Environment pane}
\protect\hypertarget{rstudio-layout-filesenvironment-pane}{}

Here you can see\ldots{}

\begin{enumerate}
\tightlist
\item
  Files in your directory (``Files'')
\item
  \texttt{Variables} in your \texttt{environment} ("Environment)
\end{enumerate}

\begin{itemize}
\tightlist
\item
  This is anything you have created in R.
\item
  Saving your work to a script allows you to recreate these variables
  again later.
\end{itemize}

\end{frame}

\begin{frame}{RStudio layout: Plots/Packages/Help/Viewer pane}
\protect\hypertarget{rstudio-layout-plotspackageshelpviewer-pane}{}

\begin{itemize}
\tightlist
\item
  This is where plots you create will show up when you call them
  (automatically in ``Plots'')
\item
  You can also\ldots{}

  \begin{itemize}
  \tightlist
  \item
    search help documentation (``Help'')
  \item
    search for packages (``Packages'')
  \end{itemize}
\end{itemize}

\end{frame}

\begin{frame}{Complete beginner?}
\protect\hypertarget{complete-beginner}{}

\begin{block}{Try this:}

\begin{enumerate}
\item
  Sign up for an account on \url{udemy.com}
\item
  Sign up for the \href{https://www.udemy.com/r-basics/}{``R basics: R
  programming language''} course on udemy
\item
  Watch videos 1, 2, 3, and 9.
\end{enumerate}

\begin{itemize}
\tightlist
\item
  1 - R basics (3 min)
\item
  2 - A walkthrough of downloading R \& Rstudio (5 min)
\item
  3 - the Rstudio interface (19 min)
\item
  9 - Three common mistakes in R beginners (11 min)
\end{itemize}

\end{block}

\end{frame}

\begin{frame}{Topics for next time}
\protect\hypertarget{topics-for-next-time}{}

My thoughts: { Basic skill } + { fun skill } per meet up

\begin{itemize}
\tightlist
\item
  { Intro to R } + { Taking notes in R Markdown }
\item
  { Cleaning data } + { making boxplots }
\item
  { Making figures in R } + { Using emoji in R }
\end{itemize}

\emph{Break up into small groups for a couple of minutes to discuss what
you would like to see at this group}

\end{frame}

\begin{frame}{Schedule for the rest of the term}
\protect\hypertarget{schedule-for-the-rest-of-the-term}{}

\begin{longtable}[]{@{}rrrr@{}}
\toprule
Date & Time & Location & Topic\tabularnewline
\midrule
\endhead
2/25 & 4pm & Cary Hall 42 & Intro to group + RStudio\tabularnewline
3/10 & 4pm & TBD & TBD\tabularnewline
3/24 & 4pm & TBD & TBD\tabularnewline
4/7 & 4pm & TBD & TBD\tabularnewline
4/21 & 4pm & TBD & TBD\tabularnewline
5/5 & 4pm & TBD & TBD\tabularnewline
\bottomrule
\end{longtable}

\end{frame}

\end{document}
