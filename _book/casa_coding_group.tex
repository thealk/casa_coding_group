% Options for packages loaded elsewhere
\PassOptionsToPackage{unicode}{hyperref}
\PassOptionsToPackage{hyphens}{url}
%
\documentclass[
]{book}
\usepackage{lmodern}
\usepackage{amssymb,amsmath}
\usepackage{ifxetex,ifluatex}
\ifnum 0\ifxetex 1\fi\ifluatex 1\fi=0 % if pdftex
  \usepackage[T1]{fontenc}
  \usepackage[utf8]{inputenc}
  \usepackage{textcomp} % provide euro and other symbols
\else % if luatex or xetex
  \usepackage{unicode-math}
  \defaultfontfeatures{Scale=MatchLowercase}
  \defaultfontfeatures[\rmfamily]{Ligatures=TeX,Scale=1}
\fi
% Use upquote if available, for straight quotes in verbatim environments
\IfFileExists{upquote.sty}{\usepackage{upquote}}{}
\IfFileExists{microtype.sty}{% use microtype if available
  \usepackage[]{microtype}
  \UseMicrotypeSet[protrusion]{basicmath} % disable protrusion for tt fonts
}{}
\makeatletter
\@ifundefined{KOMAClassName}{% if non-KOMA class
  \IfFileExists{parskip.sty}{%
    \usepackage{parskip}
  }{% else
    \setlength{\parindent}{0pt}
    \setlength{\parskip}{6pt plus 2pt minus 1pt}}
}{% if KOMA class
  \KOMAoptions{parskip=half}}
\makeatother
\usepackage{xcolor}
\IfFileExists{xurl.sty}{\usepackage{xurl}}{} % add URL line breaks if available
\IfFileExists{bookmark.sty}{\usepackage{bookmark}}{\usepackage{hyperref}}
\hypersetup{
  pdftitle={CASA Lab Coding for Speech Group},
  pdfauthor={PI: Thea Knowles, PhD},
  hidelinks,
  pdfcreator={LaTeX via pandoc}}
\urlstyle{same} % disable monospaced font for URLs
\usepackage{longtable,booktabs}
% Correct order of tables after \paragraph or \subparagraph
\usepackage{etoolbox}
\makeatletter
\patchcmd\longtable{\par}{\if@noskipsec\mbox{}\fi\par}{}{}
\makeatother
% Allow footnotes in longtable head/foot
\IfFileExists{footnotehyper.sty}{\usepackage{footnotehyper}}{\usepackage{footnote}}
\makesavenoteenv{longtable}
\usepackage{graphicx}
\makeatletter
\def\maxwidth{\ifdim\Gin@nat@width>\linewidth\linewidth\else\Gin@nat@width\fi}
\def\maxheight{\ifdim\Gin@nat@height>\textheight\textheight\else\Gin@nat@height\fi}
\makeatother
% Scale images if necessary, so that they will not overflow the page
% margins by default, and it is still possible to overwrite the defaults
% using explicit options in \includegraphics[width, height, ...]{}
\setkeys{Gin}{width=\maxwidth,height=\maxheight,keepaspectratio}
% Set default figure placement to htbp
\makeatletter
\def\fps@figure{htbp}
\makeatother
\setlength{\emergencystretch}{3em} % prevent overfull lines
\providecommand{\tightlist}{%
  \setlength{\itemsep}{0pt}\setlength{\parskip}{0pt}}
\setcounter{secnumdepth}{5}
\usepackage{booktabs}
% https://github.com/rstudio/rmarkdown/issues/337
\let\rmarkdownfootnote\footnote%
\def\footnote{\protect\rmarkdownfootnote}

% https://github.com/rstudio/rmarkdown/pull/252
\usepackage{titling}
\setlength{\droptitle}{-2em}

\pretitle{\vspace{\droptitle}\centering\huge}
\posttitle{\par}

\preauthor{\centering\large\emph}
\postauthor{\par}

\predate{\centering\large\emph}
\postdate{\par}
\usepackage[]{natbib}
\bibliographystyle{apalike}

\title{CASA Lab Coding for Speech Group}
\author{PI: Thea Knowles, PhD}
\date{}

\begin{document}
\frontmatter
\maketitle

{
\setcounter{tocdepth}{1}
\tableofcontents
}
\mainmatter
\hypertarget{welcome}{%
\chapter*{Welcome}\label{welcome}}
\addcontentsline{toc}{chapter}{Welcome}

The purpose of this group is for members and friends of the \href{https://casa-lab.netlify.com}{CASA Lab} to learn basic coding skills that are helpful for speech analysis. The languages/environments we will learn will usually be \href{https://www.r-project.org/}{\textbf{R}} and \href{http://www.fon.hum.uva.nl/praat/}{\textbf{Praat}}, though we may extend to include other languages over time. Many of the skills we will work on will be directly applicable to projects in the CASA Lab, but the hope is that you will gain skills that you will find personally valuable as well for your own work.

We will meet approximately twice a month. Meeting times TBD following an introductory meet and greet with interested members. Sometimes Thea will lead the group, and other weeks students may sign up to lead.

This is designed to be an \textbf{\emph{informal, low-stakes, hands-on group}} to help us all learn in a fun, supportive environment. Meet ups will not be mandatory, and topics will to some extent be driven by group members. There is \textbf{no expectation} of prior coding experience for you to join this group, though, if you do have experience, that's great too!

\textbf{Skills you may expect to gain:}

\begin{itemize}
\tightlist
\item
  📝 \textbf{Report generation}: How to communicate your findings by integrating your writing with your analysis code (e.g., using \href{https://bookdown.org/yihui/rmarkdown/}{R Markdown})
\item
  📊 \textbf{Data visualization}: Creating informative and aesthetically pleasing data visualizations for communicating your results (e.g., using \href{https://ggplot2.tidyverse.org/reference/ggplot.html}{\texttt{ggplot}})
\item
  📂 \textbf{File organization \& manipulation}: How to move, rename, or iterate over files on your computer.
\item
  🗣 \textbf{Acoustic analysis}: How to write scripts in \href{http://www.fon.hum.uva.nl/praat/}{Praat} for doing acoustic analysis on speech audio files.
\end{itemize}

\textbf{Equipment you need to have:}

\begin{itemize}
\tightlist
\item
  You may find it most helpful if you have \href{https://www.r-project.org/}{R} and \href{https://rstudio.com/}{R Studio} installed on your own laptop to bring to meet ups. If you don't have your own laptop, we can arrange time for you to work on the CASA lab computers.
\item
  Before each meet up, you will get an email telling you what else you need to install for that meet up, if anything.
\end{itemize}

\hypertarget{contact}{%
\section*{Contact}\label{contact}}
\addcontentsline{toc}{section}{Contact}

\textbf{Interested in joining the group?}

\begin{itemize}
\tightlist
\item
  Send an email to the PI, \href{mailto:theaknow@buffalo.edu}{Thea Knowles}.
\item
  Join our \href{https://join.slack.com/t/casa-lab-ub/shared_invite/enQtOTEyMDk1MzkxMjgzLWU3ZGVhYWViOWU1YjJkZjA0OGEzZGNkMTdhNjM4ZWM5N2FlMDEwOWI3MzE3MWQ4YzM1MmMzYmFiYWIxZGI2NTg}{Slack channel}, which we will use as a chat forum.
\end{itemize}

\href{https://casa-lab.netlify.com}{\includegraphics{img/casa-lab-logo.png}}

\hypertarget{intro}{%
\chapter{Tentative Plan \& Schedule}\label{intro}}

The schedule will depend on the goals of the group. At the moment (as of late January 2020), the tentative plan is the following:

\begin{itemize}
\tightlist
\item
  Biweekly meet ups in the CASA Lab (104 Cary Hall, UB South Campus).
\item
  Each meet up will involve a brief tutorial to the skill we'll be working on.
\item
  A new challenge will be addressed at the end, which we will go over at the next meeting.
\end{itemize}

Schedule of meet ups will appear below following the first meet up.

\emph{To appear\ldots{}}

\hypertarget{topic-1-generate-summary-reports-in-r-and-r-markdown-date-tbd}{%
\subsubsection*{Topic 1: Generate summary reports in R and R Markdown (date TBD)}\label{topic-1-generate-summary-reports-in-r-and-r-markdown-date-tbd}}
\addcontentsline{toc}{subsubsection}{Topic 1: Generate summary reports in R and R Markdown (date TBD)}

This first topic will serve as a mini introduction to R and R Studio. We will learn how to produce a short summary report using R Markdown, which allows us to integrate text and R code in the same document. These summary documents will be used as a way to keep track of our exercises and notes throughout the term, serving as a set of reference documents for us as we go along. Using R Markdown will also be useful down the line, as it can also be used to write presentations, articles, websites/blogs, and theses. For example, this website was written using R Markdown.

\hypertarget{workbook}{%
\chapter{Workbook}\label{workbook}}

This is where you will find a log of the exercises we work on.

\backmatter
  \bibliography{book.bib,packages.bib}

\end{document}
